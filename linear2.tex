\section{Linear algebra II}

% REFS: https://math.stackexchange.com/questions/433858/high-level-linear-algebra-book

\cite{Lax} has a somewhat eccentric selection of topics.

\subsection{Some affine geometry}

Affine independence. Convex sets. Simplices. Centres of simplices.

\subsection{Pick's theorem and Ehrhart polynomials}

\subsection{Inner product spaces}

\begin{definition}
Let $V$ be a real or complex vector space. An \vocab{inner product} on $V$ is a positive-definite Hermitian form on $V$.
\end{definition}

\begin{theorem}[Cauchy--Schwarz]
$|\langle x,y \rangle| \leq \| x \| \| y \|$.
\end{theorem}

An inner product $\langle -,- \rangle$ determines a norm $\| \cdot \|$, which in turn determines a distance (metric) $d(x,y) = \| x-y \|$. We also define $\cos \theta$ to make the formula $\langle x,y \rangle = \| x \| \| y \| \cos \theta$ hold (we know this is possible due to Cauchy--Schwarz).

\subsection{Orthogonality}

\begin{theorem}[Gram--Schmidt]
Every (finite-dimensional) inner product space admits an orthonormal basis.
\end{theorem}

Generalizations to infinite dimensions exist.

\begin{remark}[orthogonal polynomials]
The Gram--Schmidt algorithm, when applied to the basis $\{ 1, x, x^2, \ldots \}$ of a space of polynomials equipped with inner product $\langle f,g \rangle = \int fg \; d \alpha$ (Stieltjes) yields sequences of orthogonal polynomials. In particular when $d \alpha$ is just Lebesgue measure on $[-1,1]$ we get, up to a standardization, the \emph{Legendre polynomials}. Such ideas are quite important in applied mathematics, and we will have much more to say later on.
\end{remark}

Orthogonal complements. Orthogonal projection. Gram matrices. Moore--Penrose pseudoinverse. Polynomial interpolation. Least-squares best fit polynomials. Circumcenter of a simplex.

\begin{example}[MATH 245 midterm, Spring 2011]
Let $x_1, \ldots, x_k \in \RR^n$. Show that
\[ \argmin_{y \in \RR^n} \sum_{i=1}^k \| y - x_i \|^2 = \frac{1}{k} \sum_{i=1}^k x_i. \]
\end{example}

\begin{proof}[Solution]
Let $\Delta : \RR^n \to (\RR^n)^k$ be the diagonal map $x \mapsto (x,x,\ldots,x)$. Let $W \subset (\RR^n)^k$ be its image. Then the LHS is clearly the orthogonal projection of $(x_1,\ldots,x_k) \in (\RR^n)^k$ onto $W$.
\end{proof}

\subsection{Dual spaces and quotient spaces}

Natural isomorphism/canonical pairing $V \cong V^{**}$ given by sending $v \in V$ to the map $\mathsf{ev}_v = (\varphi \mapsto \varphi(x))$.

\subsection{Adjoint of a linear operator}

Naturality square with maps $V \to V^*$ and $W \to W^*$.

\subsection{Brief categorical remarks}

We would like here to start placing things into a more unified conceptual picture. (Discuss categories and functors, exact sequences, etc)

\subsection{Singular value decomposition}

\subsection{Canonical form theory}

\begin{remark}
In fact we will realize later this is all a special case of a basic result in module theory.
\end{remark}

\subsection{Bilinear and quadratic forms}

Classification of quadratic forms. Principal axis theorem/application of spectral theory. Sylvester's law of inertia.

\subsection{Projective spaces}

Concept of a projective space and homogeneous coordinates, Grassmannian, flag manifolds (basics), ...

\subsection{Classical matrix groups}

General linear, special linear, orthogonal, special orthogonal, unitary, special unitary, symplectic, projective versions of these. Comments on matrix groups over finite fields?

\begin{example}
What does the image of the trace map $\mathrm{SU}(n) \to \CC$ look like? It's a deltoid; John Baez has drawn pictures. (See Latex code for reference.)
% REFS: \url{https://johncarlosbaez.wordpress.com/2012/09/11/rolling-circles-and-balls-part-3}
\end{example}