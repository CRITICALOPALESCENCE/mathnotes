\section{Group theory}

\begin{definition} % FIXME: make this nicer
A \vocab{group} is a set $G$ equipped with an associative binary operation $\cdot : G \times G \to G$ for which there exists an identity element $e \in G$, and so that each $g \in G$ admits an inverse $g^{-1} \in G$.
\end{definition}

If $ab=ba$ for all $a,b \in G$ we say $G$ is \vocab{abelian} (or \vocab{commutative}).

\begin{remark}[notation]
Many types of multiplication that occur in mathematics are non-commutative (e.g.\ matrix multiplication). Addition operations, on the other hand, are always commutative. Taking a cue from this, additive notation is typically used for abelian groups: we write the operation as $+$ and write the identity as $0$, rather than $e$. In the non-abelian (or unknown) case, multiplicative notation is used: we write the operation as $\cdot$ (or simply as juxtaposition) and write the identity as $1$.
\end{remark}

\begin{example}[finite groups]
Matrix groups over finite fields, symmetric and alternating groups, braid groups, dihedral groups, cyclic groups, Coxeter groups...
\end{example}

\subsection{Cayley table}

One way to completely represent the data of a finite group is by writing down its \vocab{Cayley table} (``multiplication table'').

\begin{example}
Show some interesting colour illustrations of Cayley tables eg of $S_4$ and so on here.
% https://ww2.mathworks.cn/matlabcentral/fileexchange/46436-cayley-table-for-the-symmetric-group-s_n?s_tid=FX_rc1_behav
\end{example}

\subsection{Subgroups, quotients, homomorphisms}

The \vocab{order} of an element $a \in G$ is the smallest positive integer $n$ such that $a^n = 1$.

A subset of $G$ closed under the operations is called a \vocab{subgroup}, written $H \leq G$. The intersection of any family of subgroups is a subgroup. Each element $a \in G$ generates a subgroup, written $\langle a \rangle$. A group is called \vocab{cyclic} if it is generated by a single element.

\begin{remark}
Motivate the concept of a \vocab{normal subgroup} by \emph{trying} to define the quotient by an arbitrary subgroup and then realizing the group operation is only well-defined under a further condition. Also define \vocab{characteristic subgroup} (after automorphisms have been introduced).
\end{remark}

The subgroups of $G$ can be drawn in a lattice (Hasse) diagram.

\begin{definition}
Definition of \vocab{left/right cosets} $aH$ and $Ha$ of a subgroup $H$ in $G$. The number of cosets is called the \vocab{index}, written $[H:G]$. % or [G:H]?
\end{definition}

The group $G$ then splits into a disjoint union of left (or right) cosets.

From the coset decomposition, we immediately deduce the following important result:

\begin{theorem}[Lagrange]
If $H$ is a subgroup of a finite group $G$, then the order of $H$ divides that of $G$.
\end{theorem}

\begin{corollary}[Fermat's Little Theorem]
$a^p \cong a \bmod p$.
\end{corollary}

\begin{proof}
% FIXME: group of units notation probably hasn't been introduced yet
The order of an element $a \neq 0 \bmod p$ must divide the order of $(\mathbb{Z}/p\mathbb{Z})^\times$, which is $p-1$.
\end{proof}

\begin{corollary}
A group of prime order is cyclic.
\end{corollary}

\begin{definition}
The \vocab{center} of $G$, denoted $Z(G)$, is the set of all $a \in G$ which commute with everything.
\end{definition}

The center is a subgroup, trivially seen to be normal.

\begin{definition}
A function $f : G \to H$ is called a \vocab{group homomorphism} (or just \vocab{homomorphism} or \vocab{morphism}) if $f(ab) = f(a)f(b)$ for all $a,b \in G$, and $f(e) = e$. We then define \vocab{isomorphism}, \vocab{endomorphism}, \vocab{automorphism}, etc.
\end{definition}

The intuition is that isomorphic groups have the same algebraic structure. So if you're trying to show two groups are not isomorphic, you should find some algebraic property that one has which the other doesn't. For example, the element $2$ of $(\QQ^+, \cdot)$ cannot be written as $a^2$ for any $a \in \QQ^*$. No element of $(\RR^+, \cdot)$ has this property, so they cannot be isomorphic (you should be able to make this formal).

% Define kernel, image, cokernel, etc

\begin{example}
The following are all groups: $(\RR, +)$, $(\QQ, +)$, $(\QQ^*, \cdot)$, and so on. As an exercise, think about which are isomorphic.
\end{example}

\subsection{Isomorphism theorems}

\begin{theorem}[First isomorphism theorem]
Let $f : G \to H$ be a homomorphism of groups. Then $f$ induces an isomorphism $\tilde{f} : G/\ker f \cong \mathrm{im}(f)$.
\end{theorem}

As a result, every group homomorphism factorizes as a surjection followed by an injection (``epi-mono'').

\subsection{Simplicity}

\begin{definition}
A group is called \vocab{simple} if it has no proper nontrivial normal\footnote{Groups that have no proper nontrivial \emph{subgroups} period aren't very interesting. Why?} subgroups.
\end{definition}

In view of the first isomorphism theorem, the simple groups are those that have no nontrivial homomorphic images.

Composition series (Jordan--H\"older), etc.

Discuss the symmetry groups of Platonic solids and associated actions and isomorphisms (e.g.\ icosahedron and $A_5$).

\begin{remark}
Classification of finite simple groups was recently completed. Comments on sporadic groups here but postpone their detailed discussion.
\end{remark}

\subsection{Semidirect products, exact sequences}

\begin{definition}
Let $G$ and $H$ be groups. Then their \vocab{direct product} is the group $G \times H$ given by...
\end{definition}

Automorphisms, semidirect products (split short exact sequences), splitting lemma for exact sequences. Brief remarks on abelian categories.

\subsection{Symmetric groups}

The permutations of any set $X$ (that is, the bijections from $X$ to itself) form a group $S_X$ under composition, which we call the \vocab{symmetric group on $X$}. In particular, when $X$ is finite, say $X = \{ 1, \ldots, n \}$, we denote it simply by $S_n$ rather than $S_X$ and call it the \vocab{symmetric group of degree $n$}.

Other notations exist, such as $\mathrm{Sym}(X)$, $\mathfrak{S}_X$, etc. 

We actually saw before that $|S_n| = n!$. For $n \geq 3$, $S_n$ is nonabelian. Moreover, we have:

\begin{theorem}[Cayley]
Any group $G$ is isomorphic to a subgroup of the symmetric group $S_G$.
\end{theorem}

\begin{proof}
Just let each element $x \in G$ act on the set $G$ via $y \mapsto xy$, that is, by left multiplication. Observe that if some $y$ is fixed under the action of $x$, we must have $x=e$. Thus the action is faithful; we obtain an injective group homomorphism $G \to S_G$.
% FIXME: Group action terminology has not yet been introduced here!
\end{proof}

Subgroups of $S_n$ are called \vocab{permutation groups}.
Mention a finite presentation for $S_n$. % Leave the representation theory for later.

Since elements of $S_n$ are permutations $\sigma$ of $\{ 1, \ldots, n \}$, perhaps the most naive way of representing them is
\[ \begin{pmatrix} 1 & 2 & 3 & \cdots & n \\ \sigma(1) & \sigma(2) & \sigma(3) & \cdots & \sigma(n) \end{pmatrix}. \]
However, we can also consider an element which cyclically permutes $i_1, \ldots, i_k$ and fixes all other elements. This is typically represented as the ``cycle''
\[ (i_1 i_2 \cdots i_k). \]
For example,
\[ (1 2 3) = \begin{pmatrix} 1 & 2 & 3 \\ 2 & 3 & 1 \end{pmatrix}. \]

\subsubsection{Transpositions and parity}

\begin{definition}
A permutation that swaps two elements and fixes the rest is called a \vocab{transposition}.
\end{definition}

Clearly, transpositions are precisely those elements representable in the form $(i j)$. 

\begin{proposition}
Any $\sigma \in S_n$ can be written as a product of $m \geq 0$ transpositions. Such a representation is far from unique. However, the \emph{parity} of $m$ is an invariant of $\sigma$, which we call the \vocab{sign} and denote by $\mathrm{sgn}(\sigma)$ or $(-1)^\sigma$. It takes values in $\{ \pm 1 \}$.
\end{proposition}

We call $\sigma$ \vocab{even} or \vocab{odd} according as $\mathrm{sgn}(\sigma) = +1$ or $-1$. Thus, $\mathrm{sgn} : S_n \to \{ \pm 1 \}$ is a group homomorphism, and its kernel is (for $n \geq 2$) an index 2 normal subgroup $A_n$ of $S_n$. We call $A_n$ the \vocab{alternating group of degree $n$}.

\begin{remark}
In fact, any $\sigma$ can be written as a product of \emph{adjacent transpositions}, i.e.\ $(i j)$ with $j=i+1$. This is the idea behind bubble sort.
\end{remark}

\subsubsection{Cycle decomposition}

A \vocab{cycle} is a cyclic permutation. Any $\sigma \in S_n$ can be written as a product of disjoint cycles.

\subsubsection{Conjugacy classes}

Conjugacy classes correspond precisely to the cycle structures (partitions of $n$). (The cycle structure keeps track of the length of each cycle appearing in a disjoint cycle decomposition).

\begin{theorem}
The number of permutations with cycle structure $(1^{m_1} \cdots n^{m_n})$ is
\[ \frac{n!}{1^{m_1} \cdots n^{m_k} m_1! \cdots m_n!}. \]
Elements with this cycle structure have order LCM blah, and parity blah.
\end{theorem}

\subsubsection{Simplicity of the alternating groups}

\begin{theorem}[C.\ Jordan, 1875]
$A_n$ is simple for $n \geq 5$.
\end{theorem}

\subsubsection{Exceptional phenomena}

Brief remarks on $S_6$ and the exceptional outer automorphism, transitive subgroups of $S_n$, the anharmonic group and its relation to cross ratios and the exceptional isomorphism of $S_3$, etc.

\subsection{Structure of groups}

\begin{definition}
Let $G$ be a group, and $X$ be a set. A \vocab{(left) group action} of $G$ on $X$ is a group homomorphism $\sigma : G \to \mathrm{Sym}(X)$. Equivalently, it is a map $G \times X \to X$ satisfying some simple axioms. $G$ is said to \vocab{act} on $X$.
\end{definition}

From the definition, we can see that to each element $g \in G$ there is associated a transformation, indeed bijection, $\sigma_g = \sigma(g) : X \to X$. However, also to each element $x \in X$ we may associate the map $\lambda_x : G \to X$ given by $g \mapsto gx$.

Examples abound. 

\begin{example}
Every group acts on itself by left multiplication, conjugation, and so on.
\end{example}

If $G$ acts on a set $X$, then for each $x \in X$ we define the \vocab{orbit} of $x$ to be
\[ Gx = \{ gx \mid g \in G \}. \]
The relation of ``being in the same orbit'' is an equivalence relation, so any action induces a partition of $X$. We define the \vocab{stabilizer} (or, for physicists, the \vocab{little group}) of $x$ to be
\[ G_x = \mathrm{Stab}(x) = \{ g \in G \mid gx=x \}. \]
We call the action \vocab{transitive} if there is only a single orbit, i.e.\ $Gx=X$ for any $x \in X$. We call the action \vocab{fixed-point free} if $gx=x$ implies $g=1$ (that is, all stabilizers are trivial), and \vocab{faithful} if every group element $g \neq e$ moves \emph{some} $x \in X$ (that is, $\ker \sigma = 1$).

\begin{remark}
Many of these concepts have special names for important actions. For example, the orbit of an element under the conjugation action is called a \vocab{conjugacy class} of $G$, while a stabilizer is called a \vocab{centralizer}, denoted $C_G(x)$.
\end{remark}

\begin{remark}
In many cases of interest, $X$ will have more structure than that of a set. Often it is a geometric object, like a manifold or algebraic variety. This goes for $G$ too: it may be e.g.\ a Lie group or topological group.
\end{remark}

\begin{theorem}[orbit-stabilizer]
For any fixed $x \in X$, the map $\lambda_x : G \to X$ induces a bijection from the set $G/G_x$ of cosets of $G_x$ in $G$ to the orbit $Gx$.
\end{theorem}

\begin{proof}
The translate $gx$ depends only on the left coset $gG_x$.
\end{proof}

\begin{definition}
Let $p$ be a prime. A group $G$ is called a \vocab{$p$-group} if the order of each element is a power of $p$.
\end{definition}

\begin{example}[dihedral groups]
The \vocab{dihedral group} of order $2n$, denoted $D_n$ or $D_{2n}$, is ...
\end{example}

\begin{theorem}[Class equation?]
\[ |G| = |Z(G)| + \sum_i [G:C_G(x_i)]. \]
\end{theorem}

\subsection{Structure of abelian groups}

We now come to the central structure theorem for abelian groups\footnote{This is actually a special case of the structure theorem for finitely generated modules over a principal ideal domain ($\mathbb{Z}$ is a PID, and abelian groups are just $\mathbb{Z}$-modules).}.

\begin{theorem}[Structure Theorem]
Any finitely generated abelian group $G$ is of the form
\[ \ZZ_{p_1}^{k_1} \times \cdots \times \ZZ_{p_\ell}^{k_\ell} \times \ZZ^r \]
$G$ is finite iff $r=0$ (there is no free abelian part).
\end{theorem}

\begin{remark}
Remark on the characters of and Fourier analysis on finite abelian groups.
\end{remark}

\subsection{Advanced topics}

\subsubsection{Commutators and abelianization}

The \vocab{commutator} of $a,b \in G$ is
\[ [a,b] := aba^{-1}b^{-1}. \]
The elements commute precisely when this vanishes. We then define the \vocab{commutator subgroup} $G' = [G,G]$. 

The corresponding quotient $G/G'$ is called the \vocab{abelianization} of $G$. It has the universal property that any morphism from $G$ to an abelian group factors uniquely through $G/G'$.

\subsubsection{Nilpotency and solvability}

Discussion of lower/upper central series and nilpotency, derived series and solvability.

\subsubsection{Free groups}

Presentations (generators and relations). Free groups.

\subsection{Sylow theorems}